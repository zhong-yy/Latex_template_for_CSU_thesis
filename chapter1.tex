\chapter{绪论}
\section{使用方法}
\subsection{编译工具配置}
需要使用texlive 2020以上版本。如果用其他低版本的texlive,则需要将dissertation.tex第29行\verb|\xeCJKsetup{AutoFakeBold=2.5}|注释掉,否则会导致生成的pdf无法正常复制,影响查重。

\subsection{编译}
第一次编译需要先编译dissertation.tex文件。

\subsection{内容}
将dissertation.tex中的标题等信息替换成自己的论文信息,摘要在abstract.tex中修改,正文在chapter1.tex、chapter2.tex、chapter3.tex、chapter4.tex、chapter5.tex等文件修改。

\subsection{参考文献}
在bibfile.bib文件里面添加参考文献信息,参考文献信息可以通过百度学术、bing学术、google scholar直接导出成bib文件所需格式。

\subsection{110分钟了解\LaTeXe}
\href{https://mirrors.tuna.tsinghua.edu.cn/CTAN/info/lshort/chinese/lshort-zh-cn.pdf}{https://mirrors.tuna.tsinghua.edu.cn/CTAN/info/lshort/chinese/lshort-zh-cn.pdf}

\section{格式}
本模板按照中南大学论文格式要求定制,符合2020年、2021年格式要求。以后学校的格式要求有可能出现或大或小的修改。

模板版式如下:纸张为A4规格($297mm\times210mm$),打印区面积$240mm\times146mm$(包括篇眉),页脚距底端距离为 1.75 厘米,页眉至顶端距离为 1.5 厘米;章标题为三号加粗黑体,节标题为小四号黑体,小节标题为小四号宋体,摘要字体为四号宋体,正文字体为小四号宋体;英文字体采用Times New Roman字体;正文行间距为20pt;所用公式图表均采用自动编号,自动编号格式为X-Y,X为章号,Y为公式(图表)号。参考文献格式为国标GB/T 7714—2015 (格式会自动排版)。其他细节请查看tex源码。

\section{声明}
本模板版权归作者所有,任何人可以免费使用和转发本模板,无需通知作者。未经作者允许,任何人或组织不得将本模板用于任意形式的盈利性活动,否则追究一切责任。
作者对使用本模板造成的直接或间接损失不负任何责任。
