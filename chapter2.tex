\chapter{公式}

行内公式的书写格式为\${\kaishu 公式}\$,例如$\mathbf{F}=m\mathbf{a}$。

行间公式需要放在equation环境中,格式为
\begin{verbatim}
\begin{equation}
公式
\end{equation}
\end{verbatim}

例如,在tex文件中输入
\begin{verbatim}
\begin{equation}
\oint_{\Gamma}\mathbf{H}\cdot d\mathbf{l}
=\iint_\Omega(\mathbf{J}+\frac{d\mathbf{D}}{dt})\cdot d\mathbf{S}
\end{equation}
\end{verbatim}
运行后,会在pdf文件中显示如下公式
\begin{equation}
\oint_{\Gamma}\mathbf{H}\cdot d\mathbf{l}=\iint_\Omega(\mathbf{J}+\frac{d\mathbf{D}}{dt})\cdot d\mathbf{S}
\end{equation}

多行公式例子如下
\begin{verbatim}
\begin{equation}
\begin{split}
	a=&b+c+d+e+f+g+h+i\\
	=&\frac{x^2+y^{1/2}}{x^2+y^2}\\
	=&\sum_{i=0}^{N}x_i^2
\end{split}
\end{equation}
\end{verbatim}
产生
\begin{equation}
\begin{split}
a=&b+c+d+e+f+g+h+i\\
=&\frac{x^2+y^{1/2}}{x^2+y^2}\\
=&\sum_{i=0}^{N}x_i^2
\end{split}
\end{equation}
